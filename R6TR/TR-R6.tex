\documentclass{article}

\usepackage{amsmath}
\usepackage[T1]{fontenc}


\begin{document}
\section{Introduction}
\subsection{The problem of aerodynamic drag}
Dlaczego badamy oraz wzór na opór aerodynamiczny. Że bazujemy na modern exterior balistics i jakieś inne z literatury bo to łądnie brzmi.

\subsection{Methodology of the present work}
Solid, Ansys.

\subsection{Tested models}
R6-Endcone, R6-No-Endcone, PrawieR5

\section{Upgdated R5 model}
\begin{itemize}
    \item Domena i mesh
    \item Kolorki dla 0.2, 0.5, 0.8
    \item Wykres CD
\end{itemize}

\section{R6 Endcone}
\subsection{Solidworks}
\begin{itemize}
    \item Domena i mesh
    \item Kolorki dla 0.2, 0.5, 0.8
\end{itemize}
\subsection{Ansys Fluent with meshing}
\begin{itemize}
    \item Domena i mesh
    \item Kolorki dla 0.2, 0.5, 0.8
\end{itemize}
Wykresy obu na koniec zestawić.

\section{R6 No Endcone}
\subsection{Solidworks}
\begin{itemize}
    \item Domena i mesh
    \item Kolorki dla 0.2, 0.5, 0.8
\end{itemize}
\subsection{Ansys Fluent with meshing}
\begin{itemize}
    \item Domena i mesh
    \item Kolorki dla 0.2, 0.5, 0.8
\end{itemize}
Wykresy zestawić.

\section{Results and discussion}
\begin{itemize}
    \item Wykresy CD dla wszystkich modeli
    \item Opis porównania i zestawienie z literaturą
\end{itemize}

\end{document}